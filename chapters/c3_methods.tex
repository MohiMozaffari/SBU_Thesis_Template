\chapter{روش‌ها و مدل}
\thispagestyle{empty}
\noindent
\vskip 2cm


داده‌های این مساله را می‌توان از دو دیدگاه مدنظر قرار داده و از رهیافت‌های متفاوت بررسی کنیم:
\begin{itemize}
	\item با دید داده‌های دوبعدی و بررسی با شبکه عصبی کانولوشنالی
		\footnote{\lr{Convolutional Neural Network}} 
	\item با دید مجموعه‌هایی از سری‌های زمانی و بررسی با روش‌های رگرسیون  
\end{itemize}
در مقایسه‌ای در مورد این دو شوه متفاوت می‌توان به ویژگی‌های هر یک اشاره کرد. شبکه‌‌های عصبی کانولوشنالی امروزه بعنوان یکی از قوی‌تری ابزارها برای تحلیل داده‌های دوبعدی (عکس‌ها) شناخته شده و به کار می‌روند. 

\cite{Khadiev2019a}در مقایسه‌ای در مورد این دو شوه متفاوت می‌توان به ویژگی‌های هر یک اشاره کرد. شبکه‌‌های عصبی کانولوشنالی امروزه بعنوان یکی از قوی‌تری ابزارها برای تحلیل داده‌های دوبعدی (عکس‌ها) شناخته شده و به کار می‌روند. 
مه و مجله در ستون و سطرآنچنان که لازم است، و برای شرایط فعلی تکنولوژی مورد نیاز، و کاربردهای متنوع با هدف بهبود ابزارهای کاربردی می باشد، کتابهای زیادی در شصت و سه درصد گذشته حال و آینده، شناخت فراوان جامعه و متخصصان را می طلبد، تا با نرم افزارها شناخت بیشتری را برای طراحان رایانه ای علی الخصوص طراحان خلاقی، و فرهنگ پیشرو در زبان فارسی ایجاد کرد، در این صورت می توان امید داشت که تمام و دشواری موجود در ارائه راهکارها، و شرایط سخت تایپ به پایان رسد و زمان مورد نیاز شامل حروفچینی دستاوردهای اصلی، و جوابگوی سوالات پیوسته اهل دنیای موجود طراحی اساسا مورد استفاده قرار گی