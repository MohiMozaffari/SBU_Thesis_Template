\chapter{اصطکاک}
\label{friction}
\thispagestyle{empty}
\noindent
\vskip 2cm



\section{مساله اصطکاک}

قوانین کلاسیک مربوط به اصطکاک و آن‌چه که ما در مدرسه بعنوان قوانین اصطکاک آموزش می‌دهیم، اولین بار در قرن 18  توسط آمونتون و کولمب فرمول‌بندی شده‌اند و پیش از آنها این مساله توسط لئوناردو داوینچی مورد بررسی قرار گرفته شده است. معادله معروف نیروی اصطکاک بین دو سطح که بر روی هم کشیده می‌شود مطابق زیر نوشته می‌شود: 
\begin{equation}\label{eq:frq}
	F= \mu N
\end{equation}

\subsubsection*{اصطکاک در مقیاس نانو}

\section{میکروسکوپ نیروی اتمی}

\section{داده‌های استفاده شده}
لورم ایپسوم متن ساختگی با تولید سادگی نامفهوم از صنعت چاپ، و با استفاده از طراحان گرافیک است، چاپگرها و متون بلکه روزنامه و مجله در ستون و سطرآنچنان که لازم است، و برای شرایط فعلی تکنولوژی مورد نیاز، و کاربردهای متنوع با هدف بهبود ابزارهای کاربردی می باشد، کتابهای زیادی در شصت و سه درصد گذشته حال و آینده، شناخت فراوان جامعه و متخصصان را می طلبد، تا با نرم افزارها شناخت بیشتری را برای طراحان رایانه ای علی الخصوص طراحان خلاقی، و فرهنگ پیشرو در زبان فارسی ایجاد کرد، در این صورت می توان امید داشت که تمام و دشواری موجود در ارائه راهکارها، و شرایط سخت تایپ به پایان رسد و زمان مورد نیاز شامل حروفچینی دستاوردهای اصلی، و جوابگوی سوالات پیوسته اهل دنیای موجود طراحی اساسا مورد استفاده قرار گیرد.

