\documentclass[msc]{style/SBU-Thesis}

\setlength{\headheight}{16.65233pt}
\addtolength{\topmargin}{-4.65233pt}

\begin{document}
\pagenumbering{Alph}

% عنوان پایان‌نامه
\title{
تحلیل شبکه های مغز از روش تداوم همولوژی
}

% رشته
\subject{فیزیک }

% گرایش
\field{فیزیک آماری و سامانه‌های پیچیده }

% نام
\name{محدثه}

% نام خانوادگی
\surname{مظفری}

% استاد راهنما
\firstsupervisor{ دکتر غلامرضا جعفری }

% استاد راهنمای دوم (در صورت وجود، در غیر این صورت این خط را حذف کنید)
\secondsupervisor{دکتر سید علی حسینی اسفیدواجانی  }

%اسامی برای صفحات داوران: در هر مورد، در صورت عدم وجود، کل خط را حذف کنید
\davaranSupervisor{نام و نام‌خانوادگی}
\davaranSecondSupervisor{نام و نام‌خانوادگی}
\davaranAdvisor{نام و نام‌خانوادگی}
\davaranInternal{نام و نام‌خانوادگی}
\davaranExternal{نام و نام‌خانوادگی}
\davaranAssignee{نام و نام‌خانوادگی}
%	\davaranDate{۱۳۹۷/۱۱/۷} % چنانچه این خط را حذف کنید با فضای خالی جایگزین می‌شود

% تاریخ انجام پایان‌نامه
\thesisdate{دی ۱۴۰۳}

% نام دانشکده
\faculty{دانشکده فیزیک}

% نام دانشگاه
\university{دانشگاه شهید بهشتی}

% مقادیر انگلیسی برای صفحه آخر
\latinuniversity{Shahid Beheshti University}
\latinfaculty{Faculty of Physics}
\latintitle{Brain Networks Analysis Using Persistent Homology Technique}
\latinname{Mohaddeseh}
\latinsurname{Mozaffari}
\latinthesisdate{2025}
\firstlatinsupervisor{Dr. G. Reza Jafari}
\secondlatinsupervisor{Dr. S. Ali Hosseiny Esfidvajani}

%%%%%%%%%%%%%%%%%%%%%%%%%%%%%%%%%
% چکیده به فارسی
\fa-abstract{
این یک چکیده نمونه است که برای توصیف کلی پایان‌نامه استفاده می‌شود.
}

% کلمات کلیدی:
\keywords{
فیزیک، شبکه‌های پیچیده، توپولوژی پایدار
}

%%%%%%%%%%%%%%%%%%%%%%%%%%%%%%%%%
% ساخت صفحه اول پایان‌نامه
\firstPage 
% ساخت صفحه امضا داوران
\davaranPage 

%%%%%%%%%%%%%%%%%%%%%%%%%%%%%%%%%
%ساخت صفحه تقدیر و تشکر
{
	\newpage
	\thispagestyle{plain}
	\noindent
	\large{\textbf{با سپاس و قدردانی از}}
	
	\noindent
	پدران و مادرانی که خود را فدای تربیت فرزاندان خود کردند و\\
	اساتید و معلمانی که در تمام دوران زندگی، راهنمای جانسوز ما بودند.
	
	
	\vspace{14cm}	
	آوردن این صفحه اختیاریست.
	
	\pagebreak
}
%%%%%%%%%%%%%%%%%%%%%%%%%%%%%%%%

\rightsPage % ساخت صفحه حقوق پایان‌نامه
\copyRightPage % ساخت صفحه تعهدنامه

%%%%%%%%%%%%%%%%%%%%%%%%%%%%%%%%

%ساخت صفحه تقدیم به 
{
	\newpage
	\thispagestyle{plain}
	\large{\textbf{تقدیم به}}
	
	\begin{center}
		رهجويان علم و فناوری و دوست‌داران علم و دانش
	\end{center}
	
	\vspace{14cm}	
	آوردن این صفحه اختیاریست.
	
	\pagebreak
}

%%%%%%%%%%%%%%%%%%%%%%%%%%%%%%%%%
% پیشگفتار
\pish{
پیشگفتار نمونه‌ای که می‌توانید برای پروژه خود تنظیم کنید.
}

%%%%%%%%%%%%%%%%%%%%%%%%%%%%%%%%%
% صفحه چکیده فارسی
\abstractPage

% فهرست مطالب
\tableofcontents 

% فهرست تصاویر
\listoffigures \newpage 

% فهرست جداول
\listoftables \newpage 

% صفحه پیشگفتار
\prefacePage
		
%%%%%%%%%%%%%%%%%%%%%%%%%%%
% فصل‌های اصلی
\pagenumbering{arabic}\setcounter{page}{1}
{\baselineskip=1cm
\chapter{اصطکاک}
\label{friction}
\thispagestyle{empty}
\noindent
\vskip 2cm



\section{مساله اصطکاک}

قوانین کلاسیک مربوط به اصطکاک و آن‌چه که ما در مدرسه بعنوان قوانین اصطکاک آموزش می‌دهیم، اولین بار در قرن 18  توسط آمونتون و کولمب فرمول‌بندی شده‌اند و پیش از آنها این مساله توسط لئوناردو داوینچی مورد بررسی قرار گرفته شده است. معادله معروف نیروی اصطکاک بین دو سطح که بر روی هم کشیده می‌شود مطابق زیر نوشته می‌شود: 
\begin{equation}\label{eq:frq}
	F= \mu N
\end{equation}

\subsubsection*{اصطکاک در مقیاس نانو}

\section{میکروسکوپ نیروی اتمی}

\section{داده‌های استفاده شده}
لورم ایپسوم متن ساختگی با تولید سادگی نامفهوم از صنعت چاپ، و با استفاده از طراحان گرافیک است، چاپگرها و متون بلکه روزنامه و مجله در ستون و سطرآنچنان که لازم است، و برای شرایط فعلی تکنولوژی مورد نیاز، و کاربردهای متنوع با هدف بهبود ابزارهای کاربردی می باشد، کتابهای زیادی در شصت و سه درصد گذشته حال و آینده، شناخت فراوان جامعه و متخصصان را می طلبد، تا با نرم افزارها شناخت بیشتری را برای طراحان رایانه ای علی الخصوص طراحان خلاقی، و فرهنگ پیشرو در زبان فارسی ایجاد کرد، در این صورت می توان امید داشت که تمام و دشواری موجود در ارائه راهکارها، و شرایط سخت تایپ به پایان رسد و زمان مورد نیاز شامل حروفچینی دستاوردهای اصلی، و جوابگوی سوالات پیوسته اهل دنیای موجود طراحی اساسا مورد استفاده قرار گیرد.

}
\let\cleardoublepage\clearpage
{\baselineskip=1cm
\include{chapters/c2_backgrounds}}
\let\cleardoublepage\clearpage
{\baselineskip=1cm
\chapter{روش‌ها و مدل}
\thispagestyle{empty}
\noindent
\vskip 2cm


داده‌های این مساله را می‌توان از دو دیدگاه مدنظر قرار داده و از رهیافت‌های متفاوت بررسی کنیم:
\begin{itemize}
	\item با دید داده‌های دوبعدی و بررسی با شبکه عصبی کانولوشنالی
		\footnote{\lr{Convolutional Neural Network}} 
	\item با دید مجموعه‌هایی از سری‌های زمانی و بررسی با روش‌های رگرسیون  
\end{itemize}
در مقایسه‌ای در مورد این دو شوه متفاوت می‌توان به ویژگی‌های هر یک اشاره کرد. شبکه‌‌های عصبی کانولوشنالی امروزه بعنوان یکی از قوی‌تری ابزارها برای تحلیل داده‌های دوبعدی (عکس‌ها) شناخته شده و به کار می‌روند. 

\cite{Khadiev2019a}در مقایسه‌ای در مورد این دو شوه متفاوت می‌توان به ویژگی‌های هر یک اشاره کرد. شبکه‌‌های عصبی کانولوشنالی امروزه بعنوان یکی از قوی‌تری ابزارها برای تحلیل داده‌های دوبعدی (عکس‌ها) شناخته شده و به کار می‌روند. 
مه و مجله در ستون و سطرآنچنان که لازم است، و برای شرایط فعلی تکنولوژی مورد نیاز، و کاربردهای متنوع با هدف بهبود ابزارهای کاربردی می باشد، کتابهای زیادی در شصت و سه درصد گذشته حال و آینده، شناخت فراوان جامعه و متخصصان را می طلبد، تا با نرم افزارها شناخت بیشتری را برای طراحان رایانه ای علی الخصوص طراحان خلاقی، و فرهنگ پیشرو در زبان فارسی ایجاد کرد، در این صورت می توان امید داشت که تمام و دشواری موجود در ارائه راهکارها، و شرایط سخت تایپ به پایان رسد و زمان مورد نیاز شامل حروفچینی دستاوردهای اصلی، و جوابگوی سوالات پیوسته اهل دنیای موجود طراحی اساسا مورد استفاده قرار گی}
\let\cleardoublepage\clearpage
{\baselineskip=1cm
\include{chapters/c4_results}}
\let\cleardoublepage\clearpage

%%%%%%%%%%%%%%%%%%%%%%%%%%%
% پیوست‌ها
\appendix
\addcontentsline{toc}{part}{پیوست‌ها}

\include{appendix/a1_appendix}
\cleardoublepage

%%%%%%%%%%%%%%%%%%%%%%%%%%%
% مراجع
\newpage
\renewcommand{\bibname}{مراجع}
\bibliographystyle{ieeetr-fa}
\bibliography{library}
\bibdata{library}

%%%%%%%%%%%%%%%%%%%%%%%%%%%
% چکیده به انگلیسی
\en-abstract
{
This is an example abstract for the thesis, briefly describing the research focus and findings.
}

% کلمات کلیدی انگلیسی 
\latinkeywords
{
    Physics, Complex Networks, Persistent Homology
}

%%%%%%%%%%%%%%%%%%%%%%%%%%%
% صفحه چکیده به انگلیسی
\latinAbstractPage 

% صفحه عنوان انگلیسی
\latinFirstPage 

\end{document}
